\documentclass[a4paper,twoside=false,abstract=false,numbers=noenddot,
titlepage=false,headings=small,parskip=half,version=last]{scrartcl}

\usepackage[utf8]{inputenc}
\usepackage[T1]{fontenc}
\usepackage[english]{babel}

\usepackage[colorlinks=true, pdfstartview=FitV,
linkcolor=black, citecolor=black, urlcolor=blue]{hyperref}
\usepackage{verbatim}
\usepackage{graphicx}
\usepackage{multirow}

\usepackage{tikz}
\usetikzlibrary{matrix}

\usepackage{amsmath}
\usepackage{amsthm}
\usepackage{amssymb}
\usepackage{amsfonts}


\theoremstyle{definition}
\newtheorem{exercise}{Exercise}

\theoremstyle{remark}
\newtheorem*{solution}{Solution}
\newtheorem*{remark}{Remark}

\newtheorem{theorem}{Theorem}[section]
\newtheorem{lemma}[theorem]{Lemma}
\newtheorem{proposition}[theorem]{Proposition}
\newtheorem{corollary}[theorem]{Corollary}

\newcommand{\NN}{\ensuremath{\mathbb{N}}}
\newcommand{\ZZ}{\ensuremath{\mathbb{Z}}}
\newcommand{\QQ}{\ensuremath{\mathbb{Q}}}
\newcommand{\RR}{\ensuremath{\mathbb{R}}}
\newcommand{\CC}{\ensuremath{\mathbb{C}}}
\newcommand{\GG}{\ensuremath{\mathcal{G}}}
\newcommand{\Fourier}{\ensuremath{\mathcal{F}}}
\newcommand{\Laplace}{\ensuremath{\mathcal{L}}}

\DeclareMathOperator{\Hom}{Hom}
\DeclareMathOperator{\End}{End}
\DeclareMathOperator{\im}{im}
\DeclareMathOperator{\id}{id}

\renewcommand{\labelenumi}{(\alph{enumi})}

\author{Jim Holmström - 890503-7571}
\title{Image Based Recognition and Classification - DD2427}
\subtitle{Exercise 7}

\begin{document}

\maketitle
\thispagestyle{empty}

\begin{exercise}
{\bf
Prove the eigenvector trick
}
\end{exercise}
\begin{solution}
The respective eigenvectors and eigenvalues:
\begin{equation}
    \label{eig1}
    XX'Q=Q\Lambda
\end{equation}
\begin{equation}
    X'XQ^*=Q^*\Lambda^*
\end{equation}
Where $Q,\Lambda$ are eigenvector-matrix and eigenvalue-matrix (diagonal).
\begin{eqnarray}
    X'XQ^*&=&Q^*\Lambda^* \\
    XX'XQ^*&=&XQ^*\Lambda^* \\ 
    \label{before}
    XX'(XQ^*)&=&(XQ^*)\Lambda^* \\
    \label{conc}
    XX'\hat{Q}&=&\hat{Q}\hat{\Lambda}
\end{eqnarray}
And since Eigenvectors and eigenvalues are unique we can match \eqref{eig1} with \eqref{conc}. Knowing this relation and the renaming between \eqref{before} and \eqref{conc} we can draw the conclousion:
\begin{equation}
    XQ^*=Q,\Lambda^*=\Lambda
\end{equation}
\qed


\end{solution}
%-----------------------

\subsection{LoadData}
    \verbatiminput{../LoadData.m}

\subsection{ComputePCABasis}
    \verbatiminput{../ComputePCABasis.m}

\subsection{ReconstructFace}
    \verbatiminput{../ReconstructFace.m}

\pagebreak
\begin{figure}[t]
    \vspace{-70pt}
    \begin{center}
        \includegraphics[width=1.0\textwidth]{../Result_Pics/ADAFACES_eigen/ADA.pdf}
    \end{center}
    \vspace{-170pt}
    \caption{Eigenfaces for ADA-db}
    \label{fig:ADA}
\end{figure}
\begin{figure}[t]
    \vspace{-70pt}
    \begin{center}
        \includegraphics[width=1.0\textwidth]{../Result_Pics/BUSH_eigen/Bush.pdf}
    \end{center}
    \vspace{-170pt}
    \caption{Eigenfaces for Bush-db}
    \label{fig:Bush}
\end{figure}
\begin{figure}[t]
    \vspace{-70pt}
    \begin{center}
        \includegraphics[width=1.0\textwidth]{../Result_Pics/Reconstruct/rim.pdf}
    \end{center}
    \vspace{-170pt}
    \caption{Reconstructed faces}
    \label{fig:reconstruct}
\end{figure}




\end{document}
