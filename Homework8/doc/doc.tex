\documentclass[a4paper,twoside=false,abstract=false,numbers=noenddot,
titlepage=false,headings=small,parskip=half,version=last]{scrartcl}

\usepackage[utf8]{inputenc}
\usepackage[T1]{fontenc}
\usepackage[english]{babel}

\usepackage[colorlinks=true, pdfstartview=FitV,
linkcolor=black, citecolor=black, urlcolor=blue]{hyperref}
\usepackage{verbatim}
\usepackage{graphicx}
\usepackage{multirow}

\usepackage{tikz}
\usetikzlibrary{matrix}

\usepackage{amsmath}
\usepackage{amsthm}
\usepackage{amssymb}
\usepackage{amsfonts}
\usepackage{enumerate}


\theoremstyle{definition}
\newtheorem{exercise}{Exercise}

\theoremstyle{remark}
\newtheorem*{solution}{Solution}
\newtheorem*{remark}{Remark}

\newtheorem{theorem}{Theorem}[section]
\newtheorem{lemma}[theorem]{Lemma}
\newtheorem{proposition}[theorem]{Proposition}
\newtheorem{corollary}[theorem]{Corollary}

\newcommand{\NN}{\ensuremath{\mathbb{N}}}
\newcommand{\ZZ}{\ensuremath{\mathbb{Z}}}
\newcommand{\QQ}{\ensuremath{\mathbb{Q}}}
\newcommand{\RR}{\ensuremath{\mathbb{R}}}
\newcommand{\CC}{\ensuremath{\mathbb{C}}}
\newcommand{\GG}{\ensuremath{\mathcal{G}}}
\newcommand{\Fourier}{\ensuremath{\mathcal{F}}}
\newcommand{\Laplace}{\ensuremath{\mathcal{L}}}

\DeclareMathOperator{\Hom}{Hom}
\DeclareMathOperator{\End}{End}
\DeclareMathOperator{\im}{im}
\DeclareMathOperator{\id}{id}

\renewcommand{\labelenumi}{(\alph{enumi})}

\author{Jim Holmström - 890503-7571}
\title{Image Based Recognition and Classification - DD2427}
\subtitle{Exercise 8}

\begin{document}

\maketitle
\thispagestyle{empty}

\begin{exercise}
Considering the XOR-problem
\begin{enumerate}[i)]
    \item Are the classes lineary separable?
    \item Let the weak classifiers be defined by
        \begin{equation}
            h_v((x,y))=\text{sgn}(a_vx+c_v) ~~h_h((x,y))=\text{sgn}(a_hy+c_h)
        \end{equation}
        Work through one iteration of the boosting algorithm. What is the problem? Can we use this set of weak classifiers to solve the XOR problem?
    \item Let the weak classifiers be defined by
        \begin{equation}
            h_1((x,y))=\text{sgn}(a_1(x,y)+c_v) ~~h_2((x,y))=\text{sgn}(a_2(x,-y)+c_h)
        \end{equation}

\end{enumerate}
\end{exercise}
\begin{solution}~\\
\begin{enumerate}[i)]
    \item test
    \item test2
    \item test3
\end{enumerate}

No, since no one line can classify all the points correctly.
\end{solution}
%-----------------------

\end{document}
